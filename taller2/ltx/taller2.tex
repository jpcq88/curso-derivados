\documentclass{tufte-handout}

\usepackage[utf8]{inputenc}
\usepackage[spanish, es-tabla]{babel}
\spanishdecimal{.}

\usepackage{amsmath}
\usepackage{booktabs}
\usepackage{minted}
\usepackage{graphicx}

\graphicspath{{../img/}}

\title{Taller 2: Derivados Financieros}
\author{Juan Pablo Calle Quintero}
\date{19 de octubre de 2015}

\begin{document}
\maketitle

\section*{Punto 1}

Consideremos el que el precio de una acción en el instante $t$, $S_t$, sigue la siguiente ecuación diferencia estocástica:

\begin{equation} \label{eq:p1a_ede_st}
	\mathrm{d}S_t = \mu S_t \mathrm{d}t + \sigma S_t \mathrm{d}W
\end{equation}

\subsection*{a)}

En la Figura~\ref{fig:p1_simul_precio} se observa el resultado de una simulación del precio de la acción para un periodo de un año calendario (365 días). La volatilidad anual estimada de $\sigma = 0.1$, un retorno promedio anualizado de $\mu = 0.101$ (obtenido de los retornos promedios diarios añualizados del la serie S\&P500), una tasa libre de riesgo $r = 0.05$ y un precio inicial de la acción de $S_0 = 1930$. La simulación se obtuvo con la siguiente discretización de (\ref{eq:p1a_ede_st}):

\begin{equation} \label{eq:p1a_ecn_simul}
	S_i = S_{i-1} \mathrm{e}^{(r - \frac{1}{2}\sigma^2)\mathrm{d}t + \sigma \sqrt{\mathrm{d}t} \, \phi_i}
\end{equation}

Donde $\phi_i$ es un número aleatorio extraído de una distribución normal estándar en cada iteración $i$.

\begin{figure}[!h]
    \includegraphics[scale=0.8]{p1_simul_precio.pdf}
    \caption{Realización de una muestra del precio de la acción obtenido con simulación.}
    \label{fig:p1_simul_precio}
\end{figure}

En esta muestra aleatoria el precio de la acción al cabo de una año es $S_T=2136$. En este caso, $max(S_t - K, 0) = 136$. Si realizamos $n = 1000$ simulaciones obtenemos la Figura~\ref{fig:p1_simul_precios}. El valor esperado estimado de las simulaciones es $\hat{E}[max(S_t - K, 0)] = 93.5891$.

Por lo tanto el valor esperado de la opción en $t=0$ es:

\begin{align*}
	C(S_t,0;200,1,0.05,0.1)=\mathrm{e}^{-0.05(1)} \, \hat{E}[max(S_t - 2000, 0)] = 89.0247
\end{align*}

\begin{figure}[!h]
    \includegraphics[scale=0.9]{p1_simul_precios.pdf}
    \caption{Simulación monte carlo precio de la acción a un año (1000 iteraciones).}
    \label{fig:p1_simul_precios}
\end{figure}

\subsection*{b)}

De acuerdo con la fórmula de Black-Scholes, el precio de una opción call europea es:

\begin{equation}
	C = S_0 \, \Phi \left( \frac{(r+\frac{\sigma^2}{2})T + \mathrm{ln}(\frac{S_0}{K})}{\sigma \sqrt{T}} \right) - K \, \mathrm{e}^{-rT} \, \Phi \left( \frac{(r-\frac{\sigma^2}{2})T + \mathrm{ln}(\frac{S_0}{K})}{\sigma \sqrt{T}} \right)
\end{equation}


Remplazando los valores se obtiene el precio de la opción con el modelo Balck-Scholes. El valor de la opción call es $C_0=90.9724$, muy cercano al que se obtuvo con la simulación de monte carlo. \marginnote{Con la función GBSOption() del paquete fOptions de R se obtiene el mismo resultado.}


\subsection*{c)}

Supongamos un modelo GARCH(1,1) para los log-retornos de las serie S\&P500 de la siguiente manera:

\begin{align} \label{eq:p1c_garch}
	\sigma_t^2 = \omega + \alpha U_{t-1}^2 + \beta \sigma_{t-1}^2 \: \text{ donde  } U_t = \sigma_t \epsilon_t \text{  y  } \epsilon_t \text{ ruido blanco}
\end{align}

Si ajustamos el modelo (\ref{eq:p1c_garch}) en R con la función garcFit() del paquete fGarch, obtenemos los siguientes resultados:

\begin{align*}
	\hat{\sigma}_t^2 = 6.58\times10^{-6} + 0.1830 U_{t-1}^2 + 0.7177 \sigma_{t-1}^2
\end{align*}

\begin{marginfigure}
	\includegraphics{p1c_sigma_est.pdf}
	\caption{Volatilidad estimada del índice S\&P500}
\end{marginfigure}

La última fecha es el el dos de octubre, por lo tanto para valorar la opción el 5 de octubre necesitamos pronosticar tres periodos adelante. Recordemos que el pronóstico $l$ periodos adelante es:

\begin{align*}
	\hat{\sigma}_t^2(l) = \omega + (\alpha + \beta) \hat{\sigma}_{t+l-1}^2
\end{align*}

Los pronósticos estimados para $l=1,2,3$ son $\hat{\sigma}_t^2(1) = 0.00061$, $\hat{\sigma}_t^2(2) = 0.00069$ y $\hat{\sigma}_t^2(3) = 0.00071$. Este último es el proóstico que nos interesa. La volatilidad que nos sirve para valorar la opción el 5 de octubre es $\sqrt{\hat{\sigma}_t^2(3)} = 0.0277$. Que anualizada equivale a $0.0277\times \sqrt{252} = 0.4245$.


\section*{Punto 2}

\subsection*{a)}

Estimando la siguiente regresión en R:

\begin{align*}
	\Delta P = \beta_0 + \beta_1 \Delta F + \epsilon_t
\end{align*}

\begin{marginfigure}
	\includegraphics{p2_reg_margen.pdf}
	\caption{Regresión incrementos del protafolio vs. incrementos del futuro $\Delta P = \beta_0 + \beta_1 \Delta F + \epsilon_t$.}
\end{marginfigure}

Obtenemos un valor de $\hat{\beta}_1=R^2=\rho^2=0.3880$, por lo tanto, $\rho=\sqrt{0.3880}=0.6229$. Luego, el radio óptimo de cobertura $h^*$ es:

\begin{align*}
	h^* = \rho \frac{\sigma_P}{\sigma_F} = 0.6229 \frac{14778.17}{334.42} = 27.53
\end{align*}

Donde $\sigma_P = 14778.17$ es la volatilidad del portafolio y $\sigma_F = 334.42$ es la volatilidad del futuro.

\subsection*{b)}

El valor del portafolio es $V_P=2000\times S_{att} + 1000 \times S_{msoft} = 73400$ y el valor del futuro es $V_F=250\times 301752.5$. Por lo tanto:

\begin{align*}
	N^* = h^* \frac{V_P}{V_F} = 27.53 \frac{73400}{301752.5} = 6.7011
\end{align*}


\subsection*{c)}

Históricamente el cambio en el valor del portafolio $\Delta P$ tiene un a media $\mu_P=-312.08$ y una desviación estándar de $\sigma_P=4587.77$. Para el portafolio $\Delta X$ los valores estimados de la media y la desviación estándar son $\mu_X=-270.52$ y $\sigma_X=4371.26$. Con estos datos simulamos $N=1000$ observaciones para los dos portafolios. En la Figura~\ref{fig:p2_densidad} se muestran los gráficos de ambas densidades estimadas.

\begin{figure}[!h]
    \includegraphics[scale=0.9]{p2_densidad.pdf}
    \caption{Densidades de los portafolios $\Delta P$ y $\Delta X$ simulados suponiendo que se distribuyen normal. $N=1000$ observaciones simuladas.}
    \label{fig:p2_densidad}
\end{figure}

\section*{Punto 3}

Considere el proceso

\begin{equation} \label{eq:p3a_rt}
	r_t = r_0 \, \mathrm{e}^{(-a - \frac{\sigma^2}{2}) t + \sigma W_t}
			+ ab \, \mathrm{e}^{(-a - \frac{\sigma^2}{2}) t + \sigma W_t}
			\int_0^t
				\mathrm{e}^{(a + \frac{\sigma^2}{2}) s - \sigma W_s}
			\mathrm{d}s
\end{equation}

\subsection*{a)}

Veamos que (\ref{eq:p3a_rt}) cumple la ecuación diferencial estocástica

\begin{equation} \label{eq:p3a_drt}
	\mathrm{d}r_t = a(b-r_t)\mathrm{d}t + \sigma r_t \mathrm{d} W
\end{equation}

De acuedo con el lema de Itô,

\begin{equation} \label{eq:p3a_ito}
	\mathrm{d}r_t = \left( \frac{\partial r_t}{\partial r} a(b-r_t) +
							\frac{\partial r_t}{\partial t} +
							 r_t^2 \frac{\sigma^2}{2} \frac{\partial^2 r_t}{\partial r^2} \right) \mathrm{d}t
					+ r_t \sigma \frac{\partial r_t}{\partial r} \mathrm{d}W
\end{equation}

Remplazando (\ref{eq:p3a_rt}) en (\ref{eq:p3a_ito}) y notando que $r_t$ no depende de $r$, solo de $t$, entonces tenemos que:

\begin{align*}
	\mathrm{d}r_t =\; & \frac{\partial r_t}{\partial t} \\
				  =\; & \frac{\partial}{\partial t} \left(r_0 \, \mathrm{e}^{(-a - \frac{\sigma^2}{2}) t + \sigma W_t}
			+ ab \, \mathrm{e}^{(-a - \frac{\sigma^2}{2}) t + \sigma W_t}
			\int_0^t
				\mathrm{e}^{(a + \frac{\sigma^2}{2}) s - \sigma W_s}
			\mathrm{d}s \right) \\
			=\; & r_0 (-a \mathrm{d}t + \sigma \mathrm{d}W)\mathrm{e}^{(-a - \frac{\sigma^2}{2}) t + \sigma W_t} + ab\mathrm{e}^{(-a - \frac{\sigma^2}{2}) t + \sigma W_t} \mathrm{e}^{(a + \frac{\sigma^2}{2}) s - \sigma W_s} \mathrm{d}t \\
			& + ab (-a \mathrm{d}t + \sigma \mathrm{d}W)\mathrm{e}^{(-a - \frac{\sigma^2}{2}) t + \sigma W_t}\int_0^t
				\mathrm{e}^{(a + \frac{\sigma^2}{2}) s - \sigma W_s}
			\mathrm{d}s \\
			=\; & ab \mathrm{d}t - a r_0 \mathrm{e}^{(-a - \frac{\sigma^2}{2}) t + \sigma W_t} \mathrm{d}t + - \left[a^2 b \mathrm{e}^{(-a - \frac{\sigma^2}{2}) t + \sigma W_t}\int_0^t
				\mathrm{e}^{(a + \frac{\sigma^2}{2}) s - \sigma W_s}
			\mathrm{d}s \right] \mathrm{d}t \\
			& + \sigma \mathrm{e}^{(-a - \frac{\sigma^2}{2}) t + \sigma W_t} \mathrm{d}W + \left[ ab \sigma \mathrm{e}^{(-a - \frac{\sigma^2}{2}) t + \sigma W_t}\int_0^t
				\mathrm{e}^{(a + \frac{\sigma^2}{2}) s - \sigma W_s}
			\mathrm{d}s \right] \mathrm{d}W \\
			=\; & a(b-r_0 \, \mathrm{e}^{(-a - \frac{\sigma^2}{2}) t + \sigma W_t} - ab \, \mathrm{e}^{(-a - \frac{\sigma^2}{2}) t + \sigma W_t}
			\int_0^t
				\mathrm{e}^{(a + \frac{\sigma^2}{2}) s - \sigma W_s}
			\mathrm{d}s) \mathrm{d}t \\
			& + \sigma (r_0 \, \mathrm{e}^{(-a - \frac{\sigma^2}{2}) t + \sigma W_t}
			+ ab \, \mathrm{e}^{(-a - \frac{\sigma^2}{2}) t + \sigma W_t}
			\int_0^t
				\mathrm{e}^{(a + \frac{\sigma^2}{2}) s - \sigma W_s}
			\mathrm{d}s) \mathrm{d}W \\
			=\; & a(b-r_t)\mathrm{d}t + \sigma r_t \mathrm{d} W
\end{align*}

\subsection*{b)}

El proceso $B_t = 1000 \mathrm{e}^{r_t}$ es función de $r_t$. Supongamos que sigue un proceso de Itô de la forma:

\begin{equation} \label{eq:p3b_ito}
	\mathrm{d}B_t = \alpha(r_t,t) \mathrm{d}t + \beta(r_t,t) \mathrm{d}W
\end{equation}

Luego, por lema de Itô,

\begin{equation} \label{eq:p3b_ito2}
	\mathrm{d}B_t = \left( \frac{\partial B_t}{\partial r} \alpha(r_t,t) +
							\frac{\partial B_t}{\partial t} +
							 \beta(r_t,t)^2 \frac{\partial^2 B_t}{\partial r^2} \right) \mathrm{d}t
					+ \beta(r_t,t) B_t \frac{\partial B_t}{\partial r} \mathrm{d}W
\end{equation}

Notemos que las derivadas con respecto a $r$ son cero, ya que $r_t$ no depende de r como en eel puento anterior. Luego,

\begin{align*} \label{eq:p3b_ito2}
	\mathrm{d}B_t = & \left(\frac{\partial B_t}{\partial t}\right) \mathrm{d}t \\
				= & \frac{\partial}{\partial t} \left( exp(r_0 \, \mathrm{e}^{(-a - \frac{\sigma^2}{2}) t + \sigma W_t}
			+ ab \, \mathrm{e}^{(-a - \frac{\sigma^2}{2}) t + \sigma W_t}
			\int_0^t
				\mathrm{e}^{(a + \frac{\sigma^2}{2}) s - \sigma W_s}
			\mathrm{d}s) \right) \mathrm{d}t \\
			= & \frac{\partial r_t}{\partial t} \mathrm{e}^{r_t} \mathrm{d}t \\
			= & (a(b-r_t)dt + \sigma r_t \mathrm{d}W)\mathrm{e}^{r_t} \mathrm{d}t
\end{align*}

\pagebreak

\section*{Código de R utilizado}
% Insertar código utilizado en R
\inputminted
[
    frame=none,
    mathescape,
    fontsize=\small
]{r}{../r/taller2.R}


\end{document}
