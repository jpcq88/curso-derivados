\documentclass{tufte-handout}

\usepackage[utf8]{inputenc}
\usepackage[spanish, es-tabla]{babel}
\spanishdecimal{.}

\usepackage{amsmath}
\usepackage{booktabs}
\usepackage{minted}
\usepackage{graphicx}

\graphicspath{{../img/}}

\title{Taller 3: Derivados Financieros}
\author{Juan Pablo Calle Quintero}
\date{16 de noviembre de 2015}

\begin{document}
\maketitle

\section*{Punto 1}

\subsection*{a)}

Con la siguiente rutina del paquete RQuantLib de R se calcula la volatilidad implícita de la opción sobre el índice S\&P500, como ejemplo, el valor de la volatilidad implícita obtenido para un strike de 1.96 es 0.5233. Haciendo lo mismo para todos los valores del strike, objenemos la gráfica de la Figura~\ref{fig:p1a_grafico}.

\begin{marginfigure}
\centering
\includegraphics[scale=0.4]{p1a_grafico.pdf}
\caption{Gráfico de volatilidad impícita.}
\label{fig:p1a_grafico}
\end{marginfigure}

\begin{minted}[frame=none,mathescape,fontsize=\footnotesize]{R}
EuropeanOptionImpliedVolatility(type = 'call', value = 153.65,
                                underlying = 2102.31, strike = 2110.20,
                                riskFreeRate = 0.05, dividendYield = 0,
                                maturity = 44 / 365, volatility = 0.25)
\end{minted}

Es valor del índice el 4 de nivimbre era 2102.31, el strike es 2110.20, la tasa libre de riesgo es 5 \%, el dividendo no aplica por ser un índice, la volatilidad estimada se saca del histórico del índice y da aproximadamente 25 \%, y el tienmpo es 44 días suponiendo que empezamos el 4 de noviembre y la fecha al vencimiento es el 18 de diciembre (tercer viernes).

\subsection*{b)}

De acuerdo con los datos históricos, el retorno y la volatilidad promedio diaria es $\mu = -9.6643\times10^{-5}$ (casi cero) y $\sigma = 0.0155$ respectivamente.

Luego debemos mirar cuántas veces han caido y crecido los retornos diarios del índice por debajo o por encima de $- 3\times\sigma = -0.0466$ y $3\times\sigma = 0.0466$ respectivamente.

\begin{marginfigure}[-3cm]
\centering
\includegraphics[scale=0.4]{p1b_grafico.pdf}
\caption{Representación gráfica de los log-retornos que sobrepasan los valores críticos.}
\end{marginfigure}

Tomando solo los dos últimos años de los retornos diarios del índice S\&P500 se obtiene la Tabla~\ref{tab:p1b_frecuencias}. Se puede observar que es más frecuente que los retornos estén por debajo de tres desviaciones, casi el doble. En los últimos dos años 3.17 \% de los retornos diarios cayeron por debajo de $- 3\times\sigma$ y 1.78 \% subieron más de $3\times\sigma$.

\begin{table}
\caption{Frecuencia de los retornos por debajo y por encima de tres desviaciones estándar en los dos últimos años, 505 observaciones.}
\label{tab:p1b_frecuencias}
\centering
\begin{tabular}{lrr}
& Frecuencia & \% \\
\hline
Valores < $- 3\times\sigma$ & 16 & 3.17 \\
Valores > $3\times\sigma$ & 9 & 1.78 \\
\hline
\end{tabular}
\end{table}
 

\subsection*{c)}

Cuando se tiene una distribución de los precios del activo subyacente que es de cola izquiera más pesada y de cola derecha más liviana que la correspondiente distribución log-normal asumida, ver Figura~\ref{fig:p1c_grafico_dist}, da lugar una sonrisa de volatilidad con pendiente decreciente.

Si el precio de la opción es una función monótona creciente de la volatilidad, entonces cuando $S_T << K$ el valor de la opción va a ser mayor ya que es más probable que ocurra si la distribución real de los precios es de cola izquierda pesada, lo cual implica que la volatilidad impícita debe ser también mayor. Por otro lado, si $S_T >> K$, que dadas las caracteristicas de la distribución es menos probable respecto a una log-normal, entonces en este caso la volatilidad implícita es menor, ya que el precio de la opción debería ser más bajo.

\begin{figure}
\centering
\includegraphics[scale=0.4]{p1c_grafico_dist.pdf}
\caption{Comparativo de distribución log-normal con una de cola izquierda más pesada y cola derecha más liviana.}
\label{fig:p1c_grafico_dist}
\end{figure}

\section*{Punto 2}

\subsection*{a)}

Con la función GBSOption del paquete fOption de R se obtienen los precios $C_{K1} = 7.8967$ y $C_{K2} = 4.1837$, para un diferencial alcista de $\frac{1}{K2 - K1} (C_{K1} - C_{K2}) = 0.7426$.

\begin{minted}[frame=none,mathescape,fontsize=\footnotesize]{R}
precio_k1 <- GBSOption(TypeFlag = 'c',
                       S = S0,
                       X = K1,
                       Time = tiempo_venc,
                       r = tasa_lr,
                       b = tasa_lr,
                       sigma = sigma)@price # $C_{K1} = 7.8967$

precio_k2 <- GBSOption(TypeFlag = 'c',
                       S = S0,
                       X = K2,
                       Time = tiempo_venc,
                       r = tasa_lr,
                       b = tasa_lr,
                       sigma = sigma)@price # $C_{K2} = 4.1837$

bull_spred <- (1 / (K2 - K1)) * (precio_k1 - precio_k2) # 0.7426

delta_k1 <- GBSGreeks(Selection = 'delta',
                      TypeFlag = 'c',
                      S = S0,
                      X = K1,
                      Time = tiempo_venc,
                      r = tasa_lr,
                      b = tasa_lr,
                      sigma = sigma) # $\Delta_{K1}=0.9174$

delta_k2 <- GBSGreeks(Selection = 'delta',
                      TypeFlag = 'c',
                      S = S0,
                      X = K2,
                      Time = tiempo_venc,
                      r = tasa_lr,
                      b = tasa_lr,
                      sigma = sigma) # $\Delta_{K2}=0.7013$
\end{minted}

\subsection*{b)}

Bajo esl modelo lineal, el cambio en el portafolio $\Delta P$ está definido como:

\begin{align*}
	\Delta P = S \delta \Delta x
\end{align*}

Donde $S$ es el precio de la opción, $\delta$ es el cambio del precio de la opción respecto al precio del subyacence, y $\Delta x$ es el retorno del sbyacente en seis mese, cuyo promedio es cero y suponemos que tiene una distribución normal. Por lo tanto, desviación estándar del portafolio es:

\begin{align*}
	\sigma_{\Delta P} = \sigma_S \sqrt{\delta t} \times S_0 \times \delta= 0.3 \times \sqrt{\frac{1}{2}} \times 32 \times (0.9174 + 0.7013) = 9.5212
\end{align*}

Donde $\delta = \Delta_{K1}+\Delta_{K2}$ es la suma de las sensibilidades de cada opción en el portafolio respecto al subyacente.

Por lo tanto, el VaR por el método lineal para el portafolio (bull spread) de 1000 unidades, con una confianza del 99 \% es:

\begin{align*}
	VaR_P = -1000\times\Phi(0.01) \times \sigma_P  = -1000 \times (-2.3263) \times 9.5212 = \mathbf{\$ 22,149.67}
\end{align*}

Donde $\Phi(*)$ es la función de distribución normal inversa.

\subsection*{c)}

Bajo esl modelo cuadrático, el cambio en el portafolio $\Delta P$ está definido como:

\begin{align*}
	\Delta P = S \delta \Delta x + \frac{1}{2} S^2 \gamma (\Delta x)^2
\end{align*}


\begin{marginfigure}[2cm]
\centering
\includegraphics[scale=0.5]{p3c_densidades.pdf}
\caption{Comparación densidades $\Delta P$ bajo el modelo lineal (negro) y cuadrático (azul). Las líneas verticales son los valores del VaR al 99 \% de cada modelo.}
\label{fig:p3c_densidades}
\end{marginfigure}

En la Figura~\ref{fig:p3c_densidades} se mestran las dos densidades de $\Delta P$ bajo el modelo lineal y cuadrático usando simulación con 10 000 réplicas. Aunque las dos tienen la misma media, claramente la densidad bajo el modelo cuadrático no es simétrica, es sesgada a la derecha.

Las estimaciones de los momentos con la muestra de 10 000 réplicas son: varianza $\sigma_P^2=126.30$, exceso de kurtosis $k = 1.1764$ y un sesgo $s = 0.9094$. Con la siguiente expansión de Cornish-Fisher podemos calcular el percentil 99 de la distribución de $\Delta P$ bajo el modelo cuadrático.

\begin{align*}
	q_{0.01} = \phi_{0.01} + \frac{s}{6}(\phi_{0.01} - 1) - \frac{s^2}{36}(2\phi_{0.01}^3 - 5\phi_{0.01}) + \frac{k}{24}(\phi_{0.01}^3 - 3\phi_{0.01})
\end{align*}

Remplazando $\phi_{0.01}=-2.3263$ y los momentos estimados anteriormente obtenemos un cuantil para la distribución del portafolio $q_{0.01} = -1.6214$. Por lo tanto, el VaR para el portafolio bajo el modelo cuadrático es:

\begin{align*}
	VaR_P = -1000\times q_{0.01} \times \sigma_P  = -1000 \times (-1.6214) \times 11.2374 = \mathbf{\$ 18,220.64}
\end{align*}

Notemos que bajo este modelo el VaR es menor, y tiene sentiido, ya que la cola izquierda que es la correspondiente a las pérdidas y es menos pesada comparada con el modelo lineal.

\pagebreak
\section*{Código de R utilizado}
% Insertar código utilizado en R
\inputminted
[
    frame=none,
    mathescape,
    fontsize=\small
]{r}{../r/taller3.R}


\end{document}
