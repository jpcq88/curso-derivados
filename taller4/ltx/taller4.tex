\documentclass{tufte-handout}

\usepackage[utf8]{inputenc}
\usepackage[spanish, es-tabla]{babel}
\spanishdecimal{.}

\usepackage{amsmath}
\usepackage{booktabs}
\usepackage{minted}
\usepackage{graphicx}

\graphicspath{{../img/}}

\title{Taller 4: Derivados Financieros}
\author{Juan Pablo Calle Quintero}
\date{27 de noviembre de 2015}

\begin{document}
\maketitle

\section*{Punto 1}

\subsection*{a)}

Para los dos primero bonos, que no tienen cupón, la tasa cero es simplemente despejar $R(0,T)$ de la ecuación $100 = P(0,T) e^{T\times R(0,T)}$, con $P(0,T)$ el precio del bono al vencimiento $T$ en años. Para los otros dos casos usamos la estrategia bootstrap. Las tasas cero son:

\begin{align*}
& R(0,0.5)=ln\left(\frac{100}{98}\right) / 0.5 = 4.04 \% \\
& R(0,1)=ln \left(\frac{100}{95}\right) / 1 = 5.13 \% \\
& R(0,1.5) = ln((101 - 3.1 e^{-0.5 R(0,0.5)} - 3.1 e^{-1 R(0,1)}) / 103.1) / -1.5 = 5.44 \% \\
& R(0,2) = ln((104 - 4 e^{-0.5 R(0,0.5)} - 4 e^{-1 R(0,1)} - 4 e^{-1.5 R(0,1.5)}) / 104) / -1.5 = 5.81 \%
\end{align*}

\subsection*{b)}

En general, la tasa a plazo entre dos periodos $t$ y $T$, con $t < T$, es $f(0;t,T) = \frac{R(0,T)\times T-R(0,t)\times t}{T-t}$, luego:

\begin{align*}
& f(0;0.5,1)= \frac{R(0,1) \times 1 - R(0,0.5) \times 0.5}{1-0.5} = 6.22 \% \\
& f(0;1,1.5)= \frac{R(0,1.5) \times 1.5 - R(0,1) \times 1}{1.5-1} = 6.07 \% \\
& f(0;1.5,2) = \frac{R(0,2) \times 2 - R(0,1.5) \times 1.5}{2-1.5} = 6.91 \% 
\end{align*}

\subsection*{c)}

El precio de un bono es el valor presente de todos los flujos de dinero que recibirá quien posee el bono. El cupon es 7 \% anual, pero como se paga cada 6 meses entonces es 3.5 \% semestral, que equivale a \$ 175 ya que el principal es \$ 5.000, por lo tanto, el precio del bono es:

\begin{align*}
P(0,2) = \, & \$ 175e^{-0.5\times R(0,0.5)} + \$ 175e^{-1\times R(0,1)} + \$ 175e^{-1.5\times R(0,1.5)} + \$ 5175e^{-2\times R(0,2)} \\
 = \, & \$ 175e^{-0.5\times 4.04 \%} + \$ 175e^{-1\times 5.13 \%} + \$ 175e^{-1.5\times 5.44 \%} + \$ 5175e^{-2\times 5.81 \%} \\
= \, & \$ 5,106.45
\end{align*}

Para el rendimiento debemos hallar $y$ en la siguiente ecuación:

\begin{align*}
\$ 175e^{-0.5y} + \$ 175e^{-1 y} + \$ 175e^{-1.5 y} + \$ 5175e^{-2 y} = \$ 5,106.45
\end{align*}

Usando la función multiroot() del paquete rootSolve de R, podemos usar el algoritmo de Newton-Raphson para hallar la solución. Obtenemos un rendimiento de $\mathbf{y=5.77 \%}$.

\begin{minted}[mathescape,fontsize=\footnotesize]{R}
yield <- function(r) {
  yld <- 175 * exp(-0.5 * r) + 175 * exp(-1 * r) +
    175 * exp(-1.5 * r) + 5175 * exp(-2 * r) - 5106.45
  
  return(yld)
}

multiroot(yield, start = 0.5)$root # 0.05772341
\end{minted}


\section*{Punto 2}

La probabilidad de que un cliente retire la plata del banco en 30 días o menos está dada por la función de distribución exponencial acumulada $F(x < 30, \lambda = 1/30) = 1 - e^{-\frac{1}{30} x} = 0.6321$. Bajo la transformación cópula, el tiempo distribuído normal estándar es $N^{-1}(F(x < 30, \lambda = 1/30))=N^{-1}(0.6321)=0.3375$.

\begin{marginfigure}[-2cm]
\includegraphics[scale=0.38]{p2_dist_exp.pdf}
\caption{Distribución exponecial de los tiempos de retiro del dinero de los clientes.}
\end{marginfigure}

Por lo tanto, con una confianza del 99 \%, el máximo valor que los clientes pueden retirar en 30 días, suponiendo una correlación cópula $\rho=0.2$ es:

\begin{align*}
\$ 1,000,000 \times N\left(\frac{N^{-1}(0.6321) + \sqrt{\rho}N^{-1}(0.99)}{\sqrt{1-\rho}}\right) = \$ 938,278.6\text{ millones de pesos}
\end{align*}

\section*{Punto 3}

\subsection*{a)}



\pagebreak
\section*{Código de R utilizado}
% Insertar código utilizado en R
\inputminted
[
    frame=none,
    mathescape,
    fontsize=\small
]{r}{../r/taller4.R}


\end{document}
