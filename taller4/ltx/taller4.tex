\documentclass{tufte-handout}

\usepackage[utf8]{inputenc}
\usepackage[spanish, es-tabla]{babel}
\spanishdecimal{.}

\usepackage{amsmath}
\usepackage{booktabs}
\usepackage{minted}
\usepackage{graphicx}

\graphicspath{{../img/}}

\title{Taller 4: Derivados Financieros}
\author{Juan Pablo Calle Quintero}
\date{27 de noviembre de 2015}

\begin{document}
\maketitle

\section*{Punto 1}

\subsection*{a)}

Para los dos primero bonos, que no tienen cupón, la tasa cero es simplemente despejar $R(0,T)$ de la ecuación $100 = P(0,T) e^{T\times R(0,T)}$, con $P(0,T)$ el precio del bono al vencimiento $T$ en años. Para los otros dos casos usamos la estrategia bootstrap. Las tasas cero son:

\begin{align*}
& R(0,0.5)=ln\left(\frac{100}{98}\right) / 0.5 = 4.04 \% \\
& R(0,1)=ln \left(\frac{100}{95}\right) / 1 = 5.13 \% \\
& R(0,1.5) = ln((101 - 3.1 e^{-0.5 R(0,0.5)} - 3.1 e^{-1 R(0,1)}) / 103.1) / -1.5 = 5.44 \% \\
& R(0,2) = ln((104 - 4 e^{-0.5 R(0,0.5)} - 4 e^{-1 R(0,1)} - 4 e^{-1.5 R(0,1.5)}) / 104) / -1.5 = 5.81 \%
\end{align*}

\subsection*{b)}

En general, la tasa a plazo entre dos periodos $t$ y $T$, con $t < T$, es $f(0;t,T) = \frac{R(0,T)\times T-R(0,t)\times t}{T-t}$, luego:

\begin{align*}
& f(0;0.5,1)= \frac{R(0,1) \times 1 - R(0,0.5) \times 0.5}{1-0.5} = 6.22 \% \\
& f(0;1,1.5)= \frac{R(0,1.5) \times 1.5 - R(0,1) \times 1}{1.5-1} = 6.07 \% \\
& f(0;1.5,2) = \frac{R(0,2) \times 2 - R(0,1.5) \times 1.5}{2-1.5} = 6.91 \% 
\end{align*}

\subsection*{c)}

El precio de un bono es el valor presente de todos los flujos de dinero que recibirá quien posee el bono. El cupon es 7 \% anual, pero como se paga cada 6 meses entonces es 3.5 \% semestral, que equivale a \$ 175 ya que el principal es \$ 5.000, por lo tanto, el precio del bono es:

\begin{align*}
P(0,2) = \, & \$ 175e^{-0.5\times R(0,0.5)} + \$ 175e^{-1\times R(0,1)} + \$ 175e^{-1.5\times R(0,1.5)} + \$ 5175e^{-2\times R(0,2)} \\
 = \, & \$ 175e^{-0.5\times 4.04 \%} + \$ 175e^{-1\times 5.13 \%} + \$ 175e^{-1.5\times 5.44 \%} + \$ 5175e^{-2\times 5.81 \%} \\
= \, & \$ 5,106.45
\end{align*}

Para el rendimiento debemos hallar $y$ en la siguiente ecuación:

\begin{align*}
\$ 175e^{-0.5y} + \$ 175e^{-1 y} + \$ 175e^{-1.5 y} + \$ 5175e^{-2 y} = \$ 5,106.45
\end{align*}

Usando la función multiroot() del paquete rootSolve de R, podemos usar el algoritmo de Newton-Raphson para hallar la solución. Obtenemos un rendimiento de $\mathbf{y=5.77 \%}$.

\begin{minted}[mathescape,fontsize=\footnotesize]{R}
yield <- function(r) {
  yld <- 175 * exp(-0.5 * r) + 175 * exp(-1 * r) +
    175 * exp(-1.5 * r) + 5175 * exp(-2 * r) - 5106.45
  
  return(yld)
}

multiroot(yield, start = 0.5)$root # 0.05772341
\end{minted}


\section*{Punto 2}

La probabilidad de que un cliente retire la plata del banco en 30 días o menos está dada por la función de distribución exponencial acumulada $F(x < 30, \lambda = 1/30) = 1 - e^{-\frac{1}{30} x} = 0.6321$. Bajo la transformación cópula, el tiempo distribuído normal estándar es $N^{-1}(F(x < 30, \lambda = 1/30))=N^{-1}(0.6321)=0.3375$.

\begin{marginfigure}[-2cm]
\includegraphics[scale=0.38]{p2_dist_exp.pdf}
\caption{Distribución exponecial de los tiempos de retiro del dinero de los clientes.}
\end{marginfigure}

Por lo tanto, con una confianza del 99 \%, el máximo valor que los clientes pueden retirar en 30 días, suponiendo una correlación cópula $\rho=0.2$ es:

\begin{align*}
\$ 1,000,000 \times N\left(\frac{N^{-1}(0.6321) + \sqrt{\rho}N^{-1}(0.99)}{\sqrt{1-\rho}}\right) = \$ 938,278.6\text{ millones de pesos}
\end{align*}

\section*{Punto 3}

\subsection*{a)}

Si en $f(t;T_1,T_2) = \frac{e^{R(t,T_2)} - e^{R(t,T_1)}}{e^{R(t,T_1)}}$ hacemos $t=T_1$ entonces tenemos $f(T_1;T_1,T_2) = \frac{e^{R(T_1,T_2)} - e^{R(T_1,T_1)}}{e^{R(T_1,T_1)}}$, pero las tasa a vista en $T_1$ $R(T_1,T_1) = 0$ pues el no ha transcurrido tiempo, porlo tanto, $f(T_1;T_1,T_2) = \frac{e^{R(T_1,T_2)} - e^{0}}{e^{0}} = e^{R(T_1,T_2)}  - 1$

\subsection*{b)}

La fórmula Black-Scholes para un \textit{floorlet} es: %\footnote{http://www.mathematik.uni-muenchen.de/~filipo/ZINSMODELLE/zinsmodelle1.pdf}

\begin{align*}
Fl(t) = L(T_2-T_1) P(t, T_2) (k \Phi(-d_2(t))) - f(t;T_1,T_2) \Phi(-d_1(t))
\end{align*}

Donde $\Phi(*)$ es la distribución normal estándar acumulada, $L$ es el principal, $k$ es la tasa piso acordada para pagarse en $T_1$ hasta $T_2$, $f(t;T_1,T_2)$ es la tasa a plazo acortada en el instante $t$ para un depósito entre $T_1$ y $T_2$, y $d_1(t)$, $\sigma(t)$ es la volatilidad de la tasa a plazo $f(t;T_1,T_2)$, y $d_2(t)$ son, respectivamente:

\begin{align*}
d_1(t) = \frac{log \left(\frac{f(t;T_1, T_2}{k}\right) + \frac{1}{2} \sigma(t)^2(T_1 - t)}{\sigma(t)\sqrt{T_1-t}}
\end{align*}

\begin{align*}
d_2(t) = \frac{log \left(\frac{f(t;T_1, T_2}{k}\right) - \frac{1}{2} \sigma(t)^2(T_1 - t)}{\sigma(t)\sqrt{T_1-t}}
\end{align*}

\subsection*{c)}

La ecuación anterior en términos de un bono es:%\footnote{http://www.rstapleton.com/papers/interest-rate\%20derivatives.pdf}

\begin{align*}
Fl(t) = [B_{t, T_2} \Phi(d_1) - KB_{t,T}\Phi(d_2)](1 + k(T_2-T_1))
\end{align*}

Donde:

\begin{align*}
d_1(t) =\, & \frac{log \left(\frac{f(t;T_1, T_2}{K}\right) + \frac{1}{2} \sigma'(t)^2(T_1 - t)}{\sigma'(t)\sqrt{T_1-t}} \\
d_2(t) =\, & d_1(t) - \sigma'(t)\sqrt{T_1-t} \\
K =\, & \frac{1}{1+k(T_2-T_1)}
\end{align*}

En este caso, $B_{t, T_2}$ es el precio en $t$ de un bono que paga \$ 1 en $T_2$, $\sigma'(t)$ es la volatilidad del bono cero cupón y $K$ es el precio de ejercicio de una opción equivalente sobre el bono.


%\subsection*{d)}


\section*{Punto 4}

Como $\mu$ es la tasa esperada de crecimiento, sí tienen sentido que sea negativa, pues un camión es un activo que se deprcia con el tiempo, es decir pierde valor. La tasa esperada de crecimiento del precio del camión es $r = -(\sigma \times \lambda ) - \mu) = -0.235$. Por lo tanto, el valor esperado del precio $S_T$ del camión en 4 años es $\hat{E}(S_t) = \$ 30,000 \times e^{-0.235\times4}=\$ 11,718.83$. El valor de la opción es entonces:

\begin{align*}
C_T = e^{-rT}[\hat{E}(S_t)\Phi(d_1) - K\Phi(d_1) ]
\end{align*}

Con $d_1=\frac{ln(\hat{E}(S_t)/K) + \sigma^2T/2}{\sigma \sqrt{T}}$ y $d_2=\frac{ln(\hat{E}(S_t)/K) - \sigma^2T/2}{\sigma \sqrt{T}}$ y remplazando a $r = 0.06$, $\sigma=0.15$, $T=4$ y $K=\$ 10,000$ el precio de la opción es $\mathbf{C_T=\$ 1,832.32}$.

\pagebreak
\section*{Código de R utilizado}
% Insertar código utilizado en R
\inputminted
[
    frame=none,
    mathescape,
    fontsize=\small
]{r}{../r/taller4.R}


\end{document}
